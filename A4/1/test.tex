\section{IaaS vs. PaaS}
Wir sind beim Vergleich von PaaS zu IaaS zu dem Schluss gekommen, dass Vorteile immer stark in
Abhängigkeit des Anwenders und des Anwendungskontextes zu sehen sind. Möchte der Anwender den Fokus auf die Entwicklung und
Bereitstellung, so kann das Verwenden von PaaS die Entwicklungszeit und auch Kosten reduzieren, er
muss sich nicht um die Wartung / Koordinierung und Erweiterung der Infrastruktur (lies: virtuellen
Maschinen) kümmern. Er ist aber gleichzeitig an das Angebot und den Service des Anbieters gebunden.
\begin{table}[h]
\centering
\begin{tabular}{p{7cm} p{7cm} }
\toprule
Vorteil & Nachteil \\
\midrule
\begin{itemize}
  \item Entwickler kann sich auf Applikation konzentrieren.
  \item Keine/Wenig Kenntnisse zur Instandhaltung der Infrastruktur notwendig.
  \item Anwender muss sich nicht um CPU/Speicher kümmern. Stellte gewünschte Grenzwerte ein.
   Management auf IS-Ebene vom PaaS-Provider.
   \item Skalierung und Load-Balancing wird vereinfacht. Nur Applikationsschicht muss angepasst
   werden.
 \item Schnelles deployen einer (fertigen) Anwendung. Leichte Verknüpfung mit Release Management
 	(Bluemix und JazzHub).
\end{itemize}& 
\begin{itemize}
\item Gebunden an das Angebot des PaaS-Anbieter. Plattformen und Protokolle werden vom Anbieter
bestimmt.  Erschwert die Migration zwischen verschiedenen Anbietern.
\item Kein Einfluss auf das OS. 
\item Kein Einfluss auf aktuellste/gepatchte Versionen (Update-Politik ist von Anbieter bestimmt).
\item Keinen Einfluss auf die Infrastrukturebene. Leistung der Server/VM werden vom Anbieter bestimmt. 
\item Keinen Einfluss ob Shared oder Dedicated Maschinen. Man muss Anbieter bei Management vertrauen.
\item Es werden nur Service-Garantien gegeben (CPU, Speicher etc).
\item Bei Verwendung einer Skalierung/LB muss man sich an die Protokolle und Prozesse des Anbieters richten
\end{itemize}\\
\bottomrule
\end{tabular}
\end{table}
