\section{Bluemix und JazzHub}

Um Bluemix und JazzHub auszuprobieren haben wir wieder die Beispielanwendung \emph{bluechatter} aus Aufgabe 4 genutzt. Nach dem Einloggen mit der IBM-Identität von Bluemix kann ein GitHub-Konto mit JazzHub verknüpft werden. Ein existierendes Repository von GitHub kann dann als App-Repository zu JazzHub hinzugefügt werden. Im Bereich \enquote{Build and Deploy} kann dann im Projekt eine sogenannte Phase erstellt werden. Diese teilt sich ein einen Build- und Deploy-Schritt. Gleichzeitig kann in dieser eingestellt werden, dass diese Phase jedes Mal ausgeführt wird, sobald in einen bestimmten Branch des Repositorys eine Änderung durchgeführt wird. Die so deployte Anwendung kann unter \url{http://faprablue.mybluemix.net/} aufgerufen werden. Das zugehörige GitHub-Repository ist \url{https://github.com/CloudFapra/bluechatter}.