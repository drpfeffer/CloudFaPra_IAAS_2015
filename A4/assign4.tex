\documentclass[a4paper]{scrartcl}

\usepackage[utf8x]{inputenc} 
\usepackage[ngerman]{babel}
\usepackage[T1]{fontenc}

\usepackage{graphicx}
\usepackage{color}
\usepackage{csquotes}
% % Mathepakete
\usepackage{amsmath}
%\usepackage{amsthm}
\usepackage{amssymb}
\usepackage{amsfonts}
%\usepackage{theorem}

\newcommand{\vektor}[1]{\ensuremath{\begin{pmatrix}#1\end{pmatrix}}} 


\usepackage{listings}		% Für Programmcode einfügen

\usepackage{pdfpages}		% PDF einbinden

\usepackage{multirow}
\usepackage{booktabs}

\usepackage{lastpage}
\usepackage{scrpage2}		     % Kopf und Fußzeile
% \usepackage[all]{xy} % Für Zeichnungen (z.B. Automaten)

% Hyperlinks, die nicht markiert sind.
\usepackage[colorlinks,
pdfpagelabels,
pdfstartview = FitH,
bookmarksopen = true,
bookmarksnumbered = true,
linkcolor = black,
plainpages = false,
hypertexnames = false,
urlcolor = blue,
citecolor = black] {hyperref}

% Farben
\definecolor{orange}{rgb}{1,0.8,0.2}
\definecolor{lila}{rgb}{0.6,0,0.6}
\definecolor{green}{rgb}{0,0.6,0}
\definecolor{pink}{rgb}{1,0,1}

% Definieren neuer Farben für den Java-Quelltext
\definecolor{javaBlue}{RGB}{42,0.0,255}
\definecolor{javaGreen}{RGB}{63,127,95}
\definecolor{javaLila}{RGB}{127,0,85}
\definecolor{javaDocBlue}{RGB}{63,95,191}
\definecolor{javaDocTags}{RGB}{127,159,191}

% Farben für XML
\definecolor{forestgreen}{RGB}{34,139,34}
\definecolor{orangered}{RGB}{239,134,64}
\definecolor{darkblue}{rgb}{0.0,0.0,0.6}
\definecolor{gray}{rgb}{0.4,0.4,0.4}


%  Sonderzeichen für Mengen von Zahlen
\newcommand{\C}{{\mathbb{C}}}         % \C für C (komplexe Zahlen)
\newcommand{\Q}{{\mathbb{Q}}}         % \Q für Q (rationale Zahlen)
\newcommand{\R}{{\mathbb{R}}}         % \R für R (reelle Zahlen)
\newcommand{\Z}{{\mathbb{Z}}}         % \Z für Z (ganze Zahlen)
\newcommand{\N}{{\mathbb{N}}}         % \N für N (natürliche Zahlen)

%  Weitere Sonderzeichen
\renewcommand{\epsilon}{\varepsilon}               % anderes Epsilon


% Listings-Einstellungen

\lstset{
basicstyle=\tiny,
%keywordstyle=\color{black}\bfseries\underbar,
identifierstyle=,
commentstyle=\color{blue},
stringstyle=\ttfamily,
showstringspaces=false,
numbers=left,
rulesepcolor=\color[gray]{0.5},
texcl=true,
commentstyle=\itshape,
tabsize=4,
columns=flexible,
frame=shadowbox,
literate={ö}{{\"o}}1
         {ä}{{\"a}}1
         {ü}{{\"u}}1
         {Ö}{{\"O}}1
         {Ä}{{\"A}}1
         {Ü}{{\"U}}1
         {ß}{{\ss}}1
}

% Style für Java
\lstdefinestyle{java}{language=Java, keywordstyle=\color{javaLila}\bfseries, commentstyle=\color{javaGreen}, stringstyle=\color{javaBlue},
numbers=left,
numberstyle=\tiny,
stepnumber=1,
frame=trbl,
showstringspaces=false,
captionpos=b,
breaklines=true,
basicstyle=\rmfamily, morecomment=[s][\color{javaDocBlue}]{/**}{*/},
tabsize=2,
emph={@author, @deprecated, @param, @return, @see, @since, @throws, @version, @serial, @serialField, @serialData, @link},
emphstyle=\color{javaDocTags},
extendedchars=true}



% Style für XML
\lstdefinestyle{XML} {
    language=XML,
    extendedchars=true, 
    breaklines=true,
    breakatwhitespace=true,
    emph={},
    emphstyle=\color{red},
    basicstyle=\ttfamily,
    columns=fullflexible,
    commentstyle=\color{gray}\upshape,
    morestring=[b]",
    morecomment=[s]{<?}{?>},
    morecomment=[s][\color{forestgreen}]{<!--}{-->},
    keywordstyle=\color{orangered},
    stringstyle=\ttfamily\color{black}\normalfont,
    tagstyle=\color{darkblue}\bf,
    morekeywords={attribute,xmlns,version,type,release, name, maxOccurs},
}


%  Hier sucht man sich den gewünschten Stil für Kopf- bzw. Fußzeile aus.
%\pagestyle{empty}                    % keine Kopf- und Fußzeilen
%\pagestyle{plain}                    % nur Seitenzahlen
%\pagestyle{headings}                 % Aktivieren für Kopf- und Fußzeilen
\pagestyle{scrheadings}							% schönere Kopf und Fußzeile
\clearscrheadings 

% % % % % % % % % % % % % % %
% Zeichenpakete (Automaten) % 
% % % % % % % % % % % % % % %
\usepackage{pgf}
\usepackage{tikz}
\usepackage{tkz-berge}
\usetikzlibrary{arrows,automata}

%  Kopf und Fußzeile ---------------------------------------

\ihead{Gruppe 2}
\cfoot{Felix Ebinger (2719182)\\Thomas Reinhardt (2635525)\\Julian Ziegler (2556772)}
\ohead{23. Juni 2015}
\ifoot{Assignement 4}
\chead{Lab Course:\\
Cloud Architectures \& Management}
\ofoot{Seite \thepage \  von \pageref{LastPage}}

\begin{document}
\section{IaaS vs. PaaS}
Wir sind beim Vergleich von PaaS zu IaaS zu dem Schluss gekommen, dass Vorteile immer stark in
Abhängigkeit des Anwenders und des Anwendungskontextes zu sehen sind. Möchte der Anwender den Fokus auf die Entwicklung und
Bereitstellung, so kann das Verwenden von PaaS die Entwicklungszeit und auch Kosten reduzieren, er
muss sich nicht um die Wartung / Koordinierung und Erweiterung der Infrastruktur (lies: virtuellen
Maschinen) kümmern. Er ist aber gleichzeitig an das Angebot und den Service des Anbieters gebunden.
\begin{table}[h]
\centering
\begin{tabular}{p{7cm} p{7cm} }
\toprule
Vorteil & Nachteil \\
\midrule
\begin{itemize}
  \item Entwickler kann sich auf Applikation konzentrieren.
  \item Keine/Wenig Kenntnisse zur Instandhaltung der Infrastruktur notwendig.
  \item Anwender muss sich nicht um CPU/Speicher kümmern. Stellte gewünschte Grenzwerte ein.
   Management auf IS-Ebene vom PaaS-Provider.
   \item Skalierung und Load-Balancing wird vereinfacht. Nur Applikationsschicht muss angepasst
   werden.
 \item Schnelles deployen einer (fertigen) Anwendung. Leichte Verknüpfung mit Release Management
 	(Bluemix und JazzHub).
\end{itemize}& 
\begin{itemize}
\item Gebunden an das Angebot des PaaS-Anbieter. Plattformen und Protokolle werden vom Anbieter
bestimmt.  Erschwert die Migration zwischen verschiedenen Anbietern.
\item Kein Einfluss auf das OS. 
\item Kein Einfluss auf aktuellste/gepatchte Versionen (Update-Politik ist von Anbieter bestimmt).
\item Keinen Einfluss auf die Infrastrukturebene. Leistung der Server/VM werden vom Anbieter bestimmt. 
\item Keinen Einfluss ob Shared oder Dedicated Maschinen. Man muss Anbieter bei Management vertrauen.
\item Es werden nur Service-Garantien gegeben (CPU, Speicher etc).
\item Bei Verwendung einer Skalierung/LB muss man sich an die Protokolle und Prozesse des Anbieters richten
\end{itemize}\\
\bottomrule
\end{tabular}
\end{table}

\section{Chat und Chat-Log mittels Bluemix}

Um den Chat und den Chat-Log mittels Bluemix zu deployen muss zunächst das CloudFoundry-Tool \verb|cf| lokal heruntergeladen werden. Mit diesem Tool kann dann der Chat und der Chat-Log gemäß der Anleitung im GitHub-Repository installiert werden. In unserem Github-Repository befindet sich ein Skript (\verb|install-chat|) dafür. Im Header dieses Skripts können die Konfigurationen für BlueMix und die zu installierenden Apps geändert werden. Das Skript gibt dann am Ende die Hosts für den Chat und den Chat-Log aus.

Der Chat-Log muss nach dem Senden der ersten Chat-Nachricht mittels des Befehls \verb|cf restart <LOG-APP-NAME>| neugestartet werden um die Chats einsehen zu können.

Beim Skalieren der Chat-App muss beachtet werden, dass die einzelnen Instanzen nicht miteinander verbunden sind, sodass man bei mehr als zwei Instanzen zufällig mit einer verbunden wird. Das bedeutet insbesondere, dass man nicht zwangsläufig mit der Instanz verbunden wird, mit der der Chat-Partner verbunden wird. Dennoch werden alle Chats in derselben Datenbank gespeichert. In der Chat-Datenbank wird nicht gespeichert, aus welcher Instanz die Chat-Nachrichten stammen. Es ist so also nur mit Informationen aus der Datenbank nicht möglich, einzelne Gesprächsverläufe zu rekonstruieren.

Das Deployment der App ist nur in der US und nicht in der EU-Region möglich, da der Datenbankserver nur in der US-Region zur Verfügung steht.


\subsection{4.4}

Da die Aufgabe explizit Wordpress nennt, haben wir uns zuerst dies angesehen.
Es stellte sich heraus, dass Wordpress einst nicht ganz simpel auf Bluemix zu nutzen war, was zum einen an einer Art fehlendem, fertigen LAMP Stack lag, zum anderen an einer nicht existenten, persistenten Dateispeichermöglichkeit in Bluemix.

Dies hat sich jedoch geändert und Wordpress lässt sich seit letztem Dezember nun enorm simpel in Bluemix aufsetzen.
Es sind essentiell nur noch wenige Klicks in der grafischen Oberfläche notwendig, was wir so auch erfolgreich durchführten.
Realisiert wird dies mittels eines sogenannten \enquote{Boilerplate} Pakets, welches praktisch ein Set an Services beinhaltet.
Es werden darüberhinaus Plugins angeboten, die eine Verknüpfung von Wordpress mit beispielsweise IBM Object Storage ermöglicht, um Mediadaten persistent zu speichern.

Anschließend widmeten wir uns den beiden genannten sample Applications.
Da \enquote{todo-apps} auf den ersten Blick eine ähnliche Komplexität wie Wordpress aufzuweisen scheint (es kommt mit fertigen deployment Skripten), haben wir \enquote{bluechatter} deployed.

Dazu wurde das Script \texttt{deploy-bluechatter.sh} genutzt.
Dieses startet nach dem Login einen Redis Datenbankservice, klont das bluechatter Respository und
pusht dieses, nach einer kleinen Modifikation hinsichtlich eines Plans (der im Repo genutzte existiert nicht mehr), zu Bluemix.

Weitere Befehle, wie in 4.3, sind nicht notwendig, da die Relationen und Konfigurationen des/der Service(s) und Application(s) in der \texttt{manifest.yml} bereits definiert sind.
Dieser Ansatz der Konfiguration empfielt sich im allgemeinen für Echtwelt-Projekte, speziell wenn die Konfiguration (ver|ge)teilt werden können soll.

\section{Bluemix und JazzHub}

Um Bluemix und JazzHub auszuprobieren haben wir wieder die Beispielanwendung \emph{bluechatter} aus Aufgabe 4 genutzt. Nach dem Einloggen mit der IBM-Identität von Bluemix kann ein GitHub-Konto mit JazzHub verknüpft werden. Ein existierendes Repository von GitHub kann dann als App-Repository zu JazzHub hinzugefügt werden. Im Bereich \enquote{Build and Deploy} kann dann im Projekt eine sogenannte Phase erstellt werden. Diese teilt sich ein einen Build- und Deploy-Schritt. Gleichzeitig kann in dieser eingestellt werden, dass diese Phase jedes Mal ausgeführt wird, sobald in einen bestimmten Branch des Repositorys eine Änderung durchgeführt wird. Die so deployte Anwendung kann unter \url{http://faprablue.mybluemix.net/} aufgerufen werden. Das zugehörige GitHub-Repository ist \url{https://github.com/CloudFapra/bluechatter}.

\end{document}
