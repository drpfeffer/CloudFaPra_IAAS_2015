\section{Chat und Chat-Log mittels Bluemix}

Um den Chat und den Chat-Log mittels Bluemix zu deployen muss zunächst das Cloud-Foundry-Tool \verb|cf| lokal heruntergeladen werden und dem Pfad hinzugefügt werden um das Skript nutzen zu können. Mit diesem Tool kann dann der Chat und der Chat-Log gemäß der Anleitung im GitHub-Repository installiert werden. In unserem Github-Repository befindet sich dafür ein Skript (\verb|install-chat|). Im Header dieses Skripts können die Konfigurationen für Bluemix und die zu installierenden Apps geändert werden. Das Skript gibt am Ende die Hosts für den Chat und den Chat-Log aus.

Der Chat-Log muss nach dem Senden der ersten Chat-Nachricht mittels des Befehls \verb|cf restart <LOG-APP-NAME>| neugestartet werden um die Chats einsehen zu können. Die Chatdatenbank muss anscheinend durch das Senden der ersten Nachricht erzeugt oder installiert werden.

Beim Skalieren der Chat-App muss beachtet werden, dass die einzelnen Instanzen nicht miteinander verbunden sind, sodass man bei mehr als einer Instanz zufällig mit einer verbunden wird. Das bedeutet insbesondere, dass man nicht zwangsläufig mit der Instanz verbunden wird, mit der der Chat-Partner verbunden wurde. Dennoch werden alle Chats in derselben Datenbank gespeichert. In der Chat-Datenbank wird nicht gespeichert, aus welcher Instanz die Chat-Nachrichten stammen. Es ist so also nur mit Informationen aus der Datenbank nicht möglich, einzelne Gesprächsverläufe zu rekonstruieren.

Das Deployment der App ist nur in der US und nicht in der EU-Region möglich, da der Datenbankserver nur in der US-Region zur Verfügung steht.

Unsere deployten Anwendungen sind unter \url{http://faprachat.mybluemix.net/} und \url{http://faprachatlogs.mybluemix.net/}  zu finden.