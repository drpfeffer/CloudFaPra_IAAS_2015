\documentclass[a4paper]{scrartcl}

\usepackage[utf8x]{inputenc} 
\usepackage[ngerman]{babel}
\usepackage[T1]{fontenc}

\usepackage{graphicx}
\usepackage{color}
\usepackage{csquotes}
% % Mathepakete
\usepackage{amsmath}
%\usepackage{amsthm}
\usepackage{amssymb}
\usepackage{amsfonts}
%\usepackage{theorem}

\newcommand{\vektor}[1]{\ensuremath{\begin{pmatrix}#1\end{pmatrix}}} 


\usepackage{listings}		% Für Programmcode einfügen

\usepackage{pdfpages}		% PDF einbinden

\usepackage{multirow}
\usepackage{booktabs}

\usepackage{lastpage}
\usepackage{scrpage2}		     % Kopf und Fußzeile
% \usepackage[all]{xy} % Für Zeichnungen (z.B. Automaten)

% Hyperlinks, die nicht markiert sind.
\usepackage[colorlinks,
pdfpagelabels,
pdfstartview = FitH,
bookmarksopen = true,
bookmarksnumbered = true,
linkcolor = black,
plainpages = false,
hypertexnames = false,
urlcolor = blue,
citecolor = black] {hyperref}

% Farben
\definecolor{orange}{rgb}{1,0.8,0.2}
\definecolor{lila}{rgb}{0.6,0,0.6}
\definecolor{green}{rgb}{0,0.6,0}
\definecolor{pink}{rgb}{1,0,1}

% Definieren neuer Farben für den Java-Quelltext
\definecolor{javaBlue}{RGB}{42,0.0,255}
\definecolor{javaGreen}{RGB}{63,127,95}
\definecolor{javaLila}{RGB}{127,0,85}
\definecolor{javaDocBlue}{RGB}{63,95,191}
\definecolor{javaDocTags}{RGB}{127,159,191}

% Farben für XML
\definecolor{forestgreen}{RGB}{34,139,34}
\definecolor{orangered}{RGB}{239,134,64}
\definecolor{darkblue}{rgb}{0.0,0.0,0.6}
\definecolor{gray}{rgb}{0.4,0.4,0.4}


%  Sonderzeichen für Mengen von Zahlen
\newcommand{\C}{{\mathbb{C}}}         % \C für C (komplexe Zahlen)
\newcommand{\Q}{{\mathbb{Q}}}         % \Q für Q (rationale Zahlen)
\newcommand{\R}{{\mathbb{R}}}         % \R für R (reelle Zahlen)
\newcommand{\Z}{{\mathbb{Z}}}         % \Z für Z (ganze Zahlen)
\newcommand{\N}{{\mathbb{N}}}         % \N für N (natürliche Zahlen)

%  Weitere Sonderzeichen
\renewcommand{\epsilon}{\varepsilon}               % anderes Epsilon


% Listings-Einstellungen

\lstset{
basicstyle=\tiny,
%keywordstyle=\color{black}\bfseries\underbar,
identifierstyle=,
commentstyle=\color{blue},
stringstyle=\ttfamily,
showstringspaces=false,
numbers=left,
rulesepcolor=\color[gray]{0.5},
texcl=true,
commentstyle=\itshape,
tabsize=4,
columns=flexible,
frame=shadowbox,
literate={ö}{{\"o}}1
         {ä}{{\"a}}1
         {ü}{{\"u}}1
         {Ö}{{\"O}}1
         {Ä}{{\"A}}1
         {Ü}{{\"U}}1
         {ß}{{\ss}}1
}

% Style für Java
\lstdefinestyle{java}{language=Java, keywordstyle=\color{javaLila}\bfseries, commentstyle=\color{javaGreen}, stringstyle=\color{javaBlue},
numbers=left,
numberstyle=\tiny,
stepnumber=1,
frame=trbl,
showstringspaces=false,
captionpos=b,
breaklines=true,
basicstyle=\rmfamily, morecomment=[s][\color{javaDocBlue}]{/**}{*/},
tabsize=2,
emph={@author, @deprecated, @param, @return, @see, @since, @throws, @version, @serial, @serialField, @serialData, @link},
emphstyle=\color{javaDocTags},
extendedchars=true}



% Style für XML
\lstdefinestyle{XML} {
    language=XML,
    extendedchars=true, 
    breaklines=true,
    breakatwhitespace=true,
    emph={},
    emphstyle=\color{red},
    basicstyle=\ttfamily,
    columns=fullflexible,
    commentstyle=\color{gray}\upshape,
    morestring=[b]",
    morecomment=[s]{<?}{?>},
    morecomment=[s][\color{forestgreen}]{<!--}{-->},
    keywordstyle=\color{orangered},
    stringstyle=\ttfamily\color{black}\normalfont,
    tagstyle=\color{darkblue}\bf,
    morekeywords={attribute,xmlns,version,type,release, name, maxOccurs},
}


%  Hier sucht man sich den gewünschten Stil für Kopf- bzw. Fußzeile aus.
%\pagestyle{empty}                    % keine Kopf- und Fußzeilen
%\pagestyle{plain}                    % nur Seitenzahlen
%\pagestyle{headings}                 % Aktivieren für Kopf- und Fußzeilen
\pagestyle{scrheadings}							% schönere Kopf und Fußzeile
\clearscrheadings 

% % % % % % % % % % % % % % %
% Zeichenpakete (Automaten) % 
% % % % % % % % % % % % % % %
\usepackage{pgf}
\usepackage{tikz}
\usepackage{tkz-berge}
\usetikzlibrary{arrows,automata}

%  Kopf und Fußzeile ---------------------------------------

\ihead{Gruppe 2}
\cfoot{Felix Ebinger (2719182)\\Thomas Reinhardt (2635525)\\Julian Ziegler (2556772)}
\ohead{\today}
\ifoot{Assignement 3}
\chead{Lab Course:\\
Cloud Architectures \& Management}
\ofoot{Seite \thepage \  von \pageref{LastPage}}

\begin{document}
\section{Deploy WordPress using AWS CloudFormation}

\subsection{Example Template}
Für die Einarbeitung haben wir das Template \enquote{WordPress basic single instance} verwendet. Das Template wurde direkt aus der Beispielvorlagen sammlung per \enquote{Launch Stack} deployed. Die Initisierung erfolgt automatisch und nach einer Wartezeit ist die WordPress Instanz online.

\subsection{ClouldFormation - WordPress on two Machines}
Wir haben uns entschieden das im ersten Teil verwendete Tamplate zu modifizieren. Das Template beinhaltet berreits die instalation von WordPress und MySQL, jedoch wird beides auf einer EC2-Instanz deployed. Für diese Aufgabenstellung müssen wir den MySQL Teil auf eine neue Maschine verschieben.

Die json Liste \texttt{Resources} muss zwei Sicherheitsgruppen (WebServer und MySQL), sowie zwei Maschienen vom Typ \texttt{AWS::EC2::Instance} für WebSerer und MySQL-Server enhalten. Der WebServer Eintrag beschreibt alle notwendigen Befehle zur Installtion und Initierung des WebServers, der MySQL Eintrag beschreibt analog alle notwendigen Befehle für den MySQL-Server.

Der WebServer muss mit der internet IP-Address mit dem MySQL server verbunden werden. Dieser Schritt muss in der Konfiguration des WebServer, genauer in der wp-config ausgeführt werden.

Es kann der CloudFormation Befehl \texttt{{ "Fn::GetAtt" : [ "MySQLServer", "PrivateIp" ] }} verwendet werden. Die Funktion gibt die PrivateIp der Resource \enquote{MySQLServer} zurück. Mehr ist zur Verbindung nicht nötig.

\subsection{Start}
Für das Deployen auf CloudFormation verwenden wir das AWS-CLI. Das Skript \texttt{createEC2.sh} verwendet das Template \texttt{wordpress_mysql.json} und setzt alle notwendigen Properties. Die Installation benötigt ca. 5-10min


\subsection{Setup}

Es wird folgend angenommen, dass juju mit einer validen Amazon Umgebung
konfiguriert ist. Das bedeutet in erster Linie, dass Zugangsdaten entweder
in der Konfigurationsdatei \texttt{~/.juju/environments.yaml} angegeben, oder
die Umgebugnsvariablen \texttt{AWS_ACCESS_KEY_ID} und
\texttt{AWS_SECRET_ACCESS_KEY} in der genutzten Shell-Umgebung exportiert sind.

Zudem sollte ein \texttt{admin-secret} gesetzt sein, sofern die juju GUI
genutzt werden soll.


\subsection{Bootstrap}

Zunächst muss eine Umgebung aufgesetzt werden.
Dazu ist das Skript \texttt{bootstrap.sh} verfügbar, welches eine Amazon
Umgebung startet und die für die weitere Ausführung vorausgesetzt wird.


\subsubsection{t2 Instanzen}

Per default nutzt juju auf Amazon m1/m3 Instanzen. \\
Es ist möglich mittels z.B. \texttt{--constraints} (siehe \texttt{bootstrap.sh})
die Nutzung von t2 Instanzen zu erzwingen, was in juju 1.23 jedoch aufgrund
eines kleinen Fehlers nicht möglich ist.

Dieser Fehler wurde von unserer Gruppe behoben (https://github.com/juju/juju/commit/143e4fea),
allerdings ist der nächste stable Release, der diesen Patch enthält (1.24), 
zum jetzigen Zeitpunkt noch nicht erschienen.

Selbst kompilieren ist zwar möglich, erfordert jedoch auch ein kompilieren der
Tools (juju agent), welche juju auf die erstellten EC2 Instanzen läd.
Die erstellten Binaries müssen beim bootstrap entweder als custom 
\texttt{agent-stream} (juju Konfiguration), oder mittels \texttt{--upload-tools}
angegeben werden. \\
\texttt{build-juju-agent.sh} baut den Agenten (auf einer EC2 Instanz), um die 
benötigten Binaries zu bekommen.

Es dürfte einfacher sein auf den 1.24 Release zu warten.


\subsection{Deploy WordPress}

Das Skript \texttt{deploy-services.sh} setzt die Aufgabe 3.2 um.
Zuerst werden Services für Wordpress und MySQL gestartet, eine Beziehung beider
hinzugefügt und anschließend Wordpress der Umgebung sichtbar gemacht (Port 80).
Abschließend wird der MySQL Service durch das Hinzufügen einer Unit skaliert.

\textbf{Hinweis}: es kam bei der Ausführung der Skripte immer mal wieder vor, 
dass juju manchmal keine Verbindung zu Amazon aufbauen konnte (als würde der
Server nicht antworten). Das ist kein persistentes Problem, ein erneutes
Ausführen des Befehls führt zum Erfolg. Weshalb dies manchmal passiert, konnten
wir nicht ermitteln. In aller Regel passierte es beim deployen mehrerer
Services direkt hintereinander, aber auch ein \texttt{sleep} im Skript schafft
merkwürdigerweise keine 100\% Sicherheit.

\section{Chat und Chat-Log mittels Bluemix}

Um den Chat und den Chat-Log mittels Bluemix zu deployen muss zunächst das CloudFoundry-Tool \verb|cf| lokal heruntergeladen werden. Mit diesem Tool kann dann der Chat und der Chat-Log gemäß der Anleitung im GitHub-Repository installiert werden. In unserem Github-Repository befindet sich ein Skript (\verb|install-chat|) dafür. Im Header dieses Skripts können die Konfigurationen für BlueMix und die zu installierenden Apps geändert werden. Das Skript gibt dann am Ende die Hosts für den Chat und den Chat-Log aus.

Der Chat-Log muss nach dem Senden der ersten Chat-Nachricht mittels des Befehls \verb|cf restart <LOG-APP-NAME>| neugestartet werden um die Chats einsehen zu können.

Beim Skalieren der Chat-App muss beachtet werden, dass die einzelnen Instanzen nicht miteinander verbunden sind, sodass man bei mehr als zwei Instanzen zufällig mit einer verbunden wird. Das bedeutet insbesondere, dass man nicht zwangsläufig mit der Instanz verbunden wird, mit der der Chat-Partner verbunden wird. Dennoch werden alle Chats in derselben Datenbank gespeichert. In der Chat-Datenbank wird nicht gespeichert, aus welcher Instanz die Chat-Nachrichten stammen. Es ist so also nur mit Informationen aus der Datenbank nicht möglich, einzelne Gesprächsverläufe zu rekonstruieren.

Das Deployment der App ist nur in der US und nicht in der EU-Region möglich, da der Datenbankserver nur in der US-Region zur Verfügung steht.



\subsection{Ansätze}

Es existieren drei Mögliche Anstze um die CSAR entsprechend zu modifizieren:
\begin{enumeration}
    \item Winery: vorhandene Moodle CSAR modifizieren
    \item Winery: neue CSAR erstellen
    \item Moddle CSAR Inhalte manuell modifizieren
\end{enumeration}

Im ersten Falle lädt man die Moodle CSAR in die grafische Oberfläche von Winery.
Die Nodes die dabei geändert werden müssen sind \enquote{Moodle App} und \enquote{Moodle DB}, respektive deren Typen \enquote{MoodleApplication} und \enquote{MoodleDatabase}, sowie die zugehörige Verknüpfung \enquote{MoodleDatabaseConnection}.
Zugehörig muss der Orchestrierungsplan abgeändert (Eclipse mit BPEL Plugin).
Des weiteren müssen diverse Metadaten (Icon, Bilder, Bschreibung) der Instanzen abgeändert werden, welche der Benutzer später in der Vinothek sieht.
Die im Hintergrund werkelnden Shell-Skripte lassen sich über die Artefakte unter \enquote{Other Elements} definieren und ändern. Diese kümmern sich um das Installieren von Apache, PHP, MySQL und Moodle (jetzt: Wordpress) sowie dem Konfigurieren dieser Programme.

Der zweite Fall beginnt mit dem erstellen eines neuen Service-Templates und erfordert ein selbstständiges Erstellen sämtlicher Nodes(Types), Relationship(Types), der Topologie, des Orchestrationplan und der ArtefaktTemplates.
Natürlich müssen auch die entsprechenden Installations- und Konfigurationsskripte erstellt werden.

Der dritte Fall ist ein minimalinvasiver Ansatz: ändern des CSAR Archivs auf Dateiebene.
Das ist möglich, da sich Wordpress und Moodle von der Topologie und den Voraussetzungen (MySQL, PHP, \ldots) gleichen und der Name einer Relation für deren Funktionalität keinen Unterschied macht.
Bei diesem Ansatz ändern wir nur die tatsächliche Funktionalität in Form der Skripte ab, die sich in der CSAR Datei finden.

Die anzupassenden Skripte sind:
\begin{itemize}
    \item \texttt{scripts/MoodleApplication/install.sh}
    \item \texttt{scripts/MoodleApplication/configure.sh}
    \item \texttt{scripts/MoodleDatabase/configure.sh}
    \item \texttt{scripts/MoodleDatabaseConnection/configureDatabaseEndpoint.sh}
\end{itemize}
Änderungen an den Skripten analog zu den eingebetteten Skripten im CloudFormation Template \texttt{wordpress_mysql.json} aus Task 1.

Die anzupassenden Metadaten sind:
\begin{itemize}
    \item \texttt{SELFSERVICE-Metadata/data.xml}
    \item \texttt{SELFSERVICE-Metadata/icon.jpg}
    \item \texttt{SELFSERVICE-Metadata/image.jpg}
\end{itemize}

\textbf{Hinweis:} Die Neuerstellung des Orchestrierungsplans ist immer bei der Abänderung der Skripte und der Namen der Node/Replationship(Types) nötig, da sich diese in diesem Plan wiederfinden.


\end{document}
