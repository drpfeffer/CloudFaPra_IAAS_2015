\section{WordPress mit OpenTOSCA}

Es existieren drei mögliche Ansätze um die CSAR entsprechend zu modifizieren:
\begin{enumerate}
    \item Winery: vorhandene Moodle CSAR modifizieren
    \item Winery: neue CSAR erstellen
    \item Moddle CSAR Inhalte manuell modifizieren
\end{enumerate}

Im ersten Falle lädt man die Moodle CSAR in die grafische Oberfläche von Winery.
Die Nodes die dabei geändert werden müssen sind ,,Moodle App`` und \enquote{Moodle DB}, respektive deren Typen \enquote{MoodleApplication} und \enquote{MoodleDatabase}, sowie die zugehörige Verknüpfung \enquote{MoodleDatabaseConnection}.
Zugehörig muss der Orchestrierungsplan abgeändert (Eclipse mit BPEL Plugin).
Des weiteren müssen diverse Metadaten (Icon, Bilder, Beschreibung) der Instanzen abgeändert werden, welche der Benutzer später in der Vinothek sieht.
Die im Hintergrund werkelnden Shell-Skripte lassen sich über die Artefakte unter \enquote{Other Elements} definieren und ändern. Diese kümmern sich um die Installation von Apache, PHP, MySQL und Moodle (jetzt: Wordpress), sowie um die Konfiguration dieser Programme.

Der zweite Fall beginnt mit dem Erstellen eines neuen Service-Templates und erfordert ein selbstständiges Erstellen sämtlicher Nodes(Types), Relationship(Types), der Topologie, des Orchestrationplan und der ArtefaktTemplates.
Natürlich müssen auch die entsprechenden Installations- und Konfigurationsskripte erstellt werden.

Der dritte Fall ist ein minimalinvasiver Ansatz: ändern des CSAR Archivs auf Dateiebene.
Das ist möglich, da sich Wordpress und Moodle von der Topologie und den Voraussetzungen (MySQL, PHP, \ldots) gleichen und der Name einer Relation für deren Funktionalität keinen Unterschied macht.
Bei diesem Ansatz ändern wir nur die tatsächliche Funktionalität in Form der Skripte ab, die sich in der CSAR Datei finden.

Die anzupassenden Skripte sind:
\begin{itemize}
    \item \texttt{scripts/MoodleApplication/install.sh}
    \item \texttt{scripts/MoodleApplication/configure.sh}
    \item \texttt{scripts/MoodleDatabase/configure.sh}
    \item \texttt{scripts/MoodleDatabaseConnection/configureDatabaseEndpoint.sh}
\end{itemize}
Änderungen an den Konfigurationsskripten die Applikation und Datenbank sind analog zu den eingebetteten Skripten im CloudFormation Template \texttt{wordpress\_mysql.json} aus Task 1. Um das Installationsskript anzupassen, gibt es drei Wege. Entweder man packt das Installationsarchiv mit in das CSAR-Archiv und entpackt es im Installationsskript in das WebRoot-Verzeichnis, oder man lädt es im Skript herunter und entpackt es oder man installiert WordPress in diesem Skript mittels der Paketverwaltung.

Zusätzlich sollten die Meta-Daten angepasst werden, damit das CSAR sich korrekterweise als WordPress-CSAR zu erkennen gibt. Die anzupassenden Metadaten sind:
\begin{itemize}
    \item \texttt{SELFSERVICE-Metadata/data.xml}
    \item \texttt{SELFSERVICE-Metadata/icon.jpg}
    \item \texttt{SELFSERVICE-Metadata/image.jpg}
\end{itemize}

\textbf{Hinweis:} Die Neuerstellung des Orchestrierungsplans ist immer bei der Abänderung der Skripte und der Namen der Node/Replationship(Types) nötig, da sich diese in diesem Plan wiederfinden.
